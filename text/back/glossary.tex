\chapter{Glosario de términos}

\textbf{Backend}: es el motor de una aplicación, se encarga de realizar las funciones en segundo plano que se encargan de que la aplicación funcione.
\bigskip

\textbf{Balanceo de carga}: técnica de configuración de servidores que permite que la carga de trabajo total se reparte entre varios de ellos para que no disminuya el rendimiento general de la infraestructura.
\bigskip

\textbf{Build}: versión compilada de un programa lista para ser ejecutada.
\bigskip

\textbf{CKAN}: plataforma de código abierto para el almacenamiento de datos que permite que estos sean publicados y compartidos fácilmente.
\bigskip

\textbf{SSH (Secure SHell)}: protocolo que permite conectarse a máquinas remotas mediante conexiones seguras de red.
\bigskip

\textbf{Dataset}: conjunto de datos que representa las características de un modelo de información.
\bigskip

\textbf{Datos abiertos}: información de interés general que es publicada de forma que sea fácilmente accesible por cualquier persona o institución interesada.
\bigskip

\textbf{Desarrollo colaborativo}: metodología de desarrollo de software en el que todo el código está disponible públicamente, permitiendo que cualquier persona pueda colaborar en el proyecto. {\tt GitHub} es una de las plataforma de desarrollo colaborativo más utilizadas.
\bigskip

\textbf{Despliegue automático}: proceso que permite que una aplicación sea instalada en un infraestructura objetivo sin tener que realizar local y manualmente todos los pasos necesarios para efectuar la instalación.
\bigskip

\textbf{DevOps (DEVelopment and OPerationS)}: metodología de desarrollo ágil en la que todo el progreso es secuencial sin diferenciar entre el desarrollo y la administración del propio software.
\bigskip

\textbf{Frontend}: es la interfaz de la aplicación, es la parte de la aplicación que el usuario utiliza para comunicarse con la misma.
\bigskip

\textbf{HTML (HyperText Markup Language)}: lenguaje de marcado que se utiliza para la realización de páginas web.
\bigskip

\textbf{Integración continua}: proceso que consiste en realizar las pruebas diseñadas para una aplicación cada vez que se realizan cambios en la misma, todo esto con el fin de encontrar lo más rápidamente posiblemente posibles errores producidos en la actualización.
\bigskip

\textbf{JavaScript}: lenguaje de programación orientado a objetos interpretado que se utiliza principalmente para cargar programas desde el lado del cliente en los navegadores web.
\bigskip

\textbf{JSON (JavaScript Object Notation)}: formato de texto plano usado para el intercambio de información, independientemente del lenguaje de programación.
\bigskip

\textbf{LaTeX}: sistema de composición de documentos que permite crear textos en diferentes formatos (artículos, cartas, libros, informes...) obteniendo una alta calidad en los documentos generados.
\bigskip

\textbf{Módulo}: fragmento de un programa desarrollado para realizar una tarea específica.
\bigskip

\textbf{MongoDB}: sistema de base de datos NoSQL de código abierto orientado a documentos.
\bigskip

\textbf{Node.js}: entorno de programación asíncrono orientado a eventos con funcionamiento asíncrono basado en JavaScript utilizado principal para ejecutar programas desde el lado del servidor.
\bigskip

\textbf{NoSQL}: sistema de gestión de base de datos que usan diversas estructuras para organizar los datos (documentos, grafos, pares clave/valor, orientación a objetos...) cuya principal ventaja es el gran rendimiento en la realización de tareas de obtención y almacenamiento de datos.
\bigskip

\textbf{NPM}: sistema de gestión de paquetes usado por {\tt Node.js}.
\bigskip

\textbf{Portal de transparencia}: sitio web cuya función es la publicación de datos abiertos.
\bigskip

\textbf{Programación asíncrona}: paradigma de programación en el que las operaciones pueden ejecutarse de forma independiente a la secuencia en la que son llamadas.
\bigskip

\textbf{Provisionamiento}: proceso que se encarga de preparar una infraestructura con todos los recursos software necesarios para poder desplegar una aplicación en ella.
\bigskip

\textbf{\textit{Pull}}: acción consistente en obtener los cambios de un proyecto desde su repositorio de {\tt GitHub}.
\bigskip

\textbf{\textit{Pull request}}: acción consistente en fusionar los cambios realizados en un repositorio desde otro repositorio que hubiera hecho una copia del proyecto con el propio repositorio original.
\bigskip

\textbf{\textit{Push}}: acción consiste en publicar los cambios hechos en un proyecto a su repositorio de {\tt GitHub}.
\bigskip

\textbf{RSA}: sistema criptográfico de clave pública usado para la seguridad de transferencia de datos.
\bigskip

\textbf{Sistema de control de versiones}: aplicación que permite gestionar los cambios que se producen durante el desarrollo de un proyecto pudiendo así llevar un histórico de los mismos, ver las diferencias introducidas ó deshacer dichos cambios, entre otras operaciones. {\tt Git} es un ejemplo de sistema de control de versiones y uno de los más usados en la actualidad.
\bigskip

\textbf{Software libre}: software cuya licencia permite que este sea usado, copiado, modificado y distribuido libremente según el tipo de licencia que adopte.
\bigskip

\textbf{TDD (Test-Driven Development)}: metodología de desarrollo de software en el que durante la fase inicial se desarrollan las pruebas que el software debe superar y posteriormente de desarrolla el software para que pase dichas pruebas.
\bigskip

\textbf{Test de cobertura}: comprobación de los resultados obtenidos por los tests unitarios que permiten conocer el porcentaje total del código del software que esta cubierto por algún tipo de prueba.
\bigskip

\textbf{Tests unitarios}: prueba de cada una de las funcionalidades implementadas en una aplicación destinadas a comprobar que la funcionalidad se desempeña correctamente.
\bigskip

\textbf{URL (Uniform Resource Locator)}: nombre y con un formato estándar que permite acceder a un recurso de forma inequívoca.
\bigskip

\textbf{YAML (YAML Ain't Another Markup Language)}: formato de serialización de datos legible por humanos cuyo principal uso es el intercambio de información, aunque también es usado como formato de archivos de configuración.