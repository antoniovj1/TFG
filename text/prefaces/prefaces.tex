\begin{center}
{\LARGE\bfseries\titulo}\\
\end{center}
\begin{center}
\autor\
\end{center}

\section*{Resumen}

\bigskip
\noindent{\textbf{Palabras clave}: \textit{software libre}, \textit{transparencia}, \textit{datos abiertos}, \textit{sistema de control de versiones}, \textit{provisionamiento}, \textit{tests}, \textit{integración continua}, \textit{despliegue automático}\\

Se pretende desarrollar una plataforma para la \textit{transparencia} que sea desplegable en un infraestructura física o virtual basándose en el trabajo realizado en el portal {\tt UGR Transparente}, respaldada por los \textit{datos abiertos} publicados en la plataforma {\tt OpenData UGR}.

\bigskip
Este desarrollo tendrá como objetivo que el resultado sea una plataforma totalmente basada en el \textit{software libre} que se pudiera adaptar con facilidad a otro organismo, teniendo en cuenta funcionalidades y requisitos obligatorios, además de aspectos de accesibilidad y escalabilidad. 

\bigskip
La herramientas básicas a usar serán un \textit{sistema de control de versiones} y un sistema de desarrollo colaborativo que albergue el proyecto, a que además use dicho \textit{sistema de control de versiones}. También se quiere que la plataforma se pueda desarrollar y administrar simultáneamente, por lo que para convertir todo esto es un proceso más ágil e ininterrumpido se usarán otras herramientas que permitan lo siguiente:

\begin{itemize}
  \item Realizar \textit{provisionamiento} de las infraestructuras.
  \item Validación mediante \textit{tests} unitarios.
  \item Comprobación de conflictos mediante \textit{integración continua}.
  \item Actualizaciones mediante \textit{despliegue automático}.
\end{itemize}

\newpage
\begin{center}
{\LARGE\bfseries\tituloEng}\\
\end{center}
\begin{center}
\autor\
\end{center}

\section*{Extended abstract}

\bigskip
\noindent{\textbf{Keywords}: \textit{free software}, \textit{transparency}, \textit{open data}, \textit{version control system}, \textit{provisioning}, \textit{tests}, \textit{continuous integration}, \textit{automated deployment}\\

Our intention is to develop a platform for \textit{transparency} that is deployable on a physical or virtual infrastructure based on work done on the site {\tt UGR Transparente}, backed by \textit{open data} published in the {\tt Open Data UGR}.

\bigskip
This development aims that the result is a platform completely based on \textit{free software} that could be easily adapted to another organization, taking into consideration features and mandatory requirements, as well as issues of accessibility and scalability. The platform which presents the data will develop in Node.js, which is a programming environment that operates at runtime based on the {\tt Google's V8 JavaScript engine}; also will use {\tt Express} for web application development and Jade to generate \textit{HTML} files based on templates. Moreover, the platform that contains the data is based on {\tt CKAN}, a open-source data portal platform
developed by \textbf {The Open Knowledge Foundation}, a not-for-profit organisation that promotes creating and sharing knowledge freely.

\bigskip
Being \textit{free software}, this development is not limited to the working group that started the work, but rather is focused on that anyone can make their contribution to the project; therefore, for all this can be handled is necessary a \textit{version control system}, in this case {\tt Git} will be used, which is practically a standard in this area. In addition, it will also be used a recognized and open collaborative based development platform as is {\tt GitHub}, which is also integrated with Git, providing ease in the development and distribution of work, because any who access the project repository can freely copy it for its open license.

\bigskip
An important point to facilitate system administration is the \textit{provisioning}. \textit{Provisioning} a machine, as the name suggests, is to provide to the machine all the resources needed for its performance, in our case we refer to all the software necessary for that the platform developed works properly in such infrastructure. In the case of \textit{transparency} portal, as the entire project is hosted on {\tt GitHub}, we can set up a utility like Ansible to download the project and then install on the machine all the necessary software so that the platform can function properly. In contrast, such as the \textit{open data} portal is not a own development, their \textit{provisioning} will consist of installing and customizing {\tt CKAN} for the \textbf {University of Granada}.

\bigskip
During the development of any software application, new features are introduced to the software gradually, so must be ensured that the new features do not compromise the overall stability of the project. For this purpose the unit \textit{tests} are written, which could be considered as small programs within the system that handles check by assertions and behavior patterns that all elements operate as they should. To verify that our unit \textit{tests} are sufficient also need to pass a coverage test that tells us that all the functionality of our platform are properly validated by the corresponding unit test. There are many libraries that allow both actions, but for the facility of work between them, unit test will be passed with {\tt Mocha} and coverage test will be with {\tt Istanbul}.

\bigskip
Once we have the \textit{tests} written, we don't need concern ourselves with run them manually, we have the option of running in a external platform every time we make changes and directly get the results, this is \textit{continuous integration}. \textit{Continuous integration} will be made with {\tt Travis CI} and the operation is very simple: every time we make a change in {\tt GitHub}, {\tt Travis CI} download the latest version of the code, builds it and passes the \textit{tests} we have written, to finish by returning the results of the \textit{tests} that will make us know whether changes have produced a conflict in the system. 

\bigskip
The last thing to keep in mind is about deploying the changes automatically on our server, can be a tedious procedure having to manually access to the infrastructure of our platform every time that we want to apply the new changes we have made in the application, so we'll use the tool {\tt Flightplan} for \textit{automated deployment}; only with specify our server as the target we can set a series of tasks that will make that automatically for the updates we're making become effective on the main machine.

\bigskip
By using these tools we are getting an application in which development and management are closely related and automated, which makes the application much more reliable and easy recovery in case of problems.

\newpage
\thispagestyle{empty}
\
\vspace{3cm}

\noindent\rule[-1ex]{\textwidth}{2pt}\\[4.5ex]

Yo, \textbf{\autor}, alumno de la titulación \textbf{\grado} de la \textbf{\escuela\ de la \universidad}, autorizo la ubicación de la siguiente copia de mi Trabajo Fin de Grado (\textit{\titulo}) en la biblioteca del centro para que pueda ser consultada por las personas que lo deseen.

\bigskip
Además, este mismo trabajo es realizado bajo licencia \textbf{Creative Commons Attribution-ShareAlike 4.0} (\url{https://creativecommons.org/licenses/by-sa/4.0/}), dando permiso para copiarlo y redistribuirlo en cualquier medio o formato, también de adaptarlo de la forma que se quiera, pero todo esto siempre y cuando se reconozca la autoría y se distribuya con la misma licencia que el trabajo original. El documento en formato {\tt LaTeX} se puede encontrar en el siguiente repositorio de {\tt GitHub}: \url{https://github.com/germaaan/TFG}.

\vspace{4cm}

\noindent Fdo: \autor

\vspace{2cm}

\begin{flushright}
\ciudad, a \today
\end{flushright}

\newpage
\thispagestyle{empty}
\
\vspace{3cm}

\noindent\rule[-1ex]{\textwidth}{2pt}\\[4.5ex]

D. \textbf{\tutor}, profesor del \textbf{Departamento de Arquitectura y Tecnología de Computadores} de la \textbf{\universidad}.

\vspace{0.5cm}

\vspace{0.5cm}

\textbf{Informa:}

\vspace{0.5cm}

Que el presente trabajo, titulado \textit{\textbf{\titulo}}, ha sido realizado bajo su supervisión por \textbf{\autor}, y 
autoriza la defensa de dicho trabajo ante el tribunal que corresponda.

\vspace{0.5cm}

Y para que conste, expide y firma el presente informe en \ciudad\ a \today.

\vspace{1cm}

\textbf{El tutor:}

\vspace{5cm}

\noindent \textbf{\tutor}

\chapter*{Agradecimientos}
\thispagestyle{empty}

\vspace{1cm}

A mi familia, porque en mayor o menor medida les debo el qué y el cómo soy hoy en día.

\bigskip
A mis amigos cercanos, porque en muchas ocasiones y sin darse cuenta, son los que hacen que no me rinda y siga recordando quién soy yo.

\bigskip
A mis compañeros de informática (que muchos ya pertenecen al grupo anterior), porque juntos no hay quién nos pare cuando nos ponemos a hacer lo que sea.

\bigskip
A mis compañeros de la OSL, porque han hecho que mi primera experiencia laboral (durante la que he desarrollado este proyecto) vaya a ser difícilmente superable por las siguientes.