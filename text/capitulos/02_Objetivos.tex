\chapter{Objetivos}

El objetivo de este proyecto es el de obtener un portal de transparencia para la \textbf{Universidad de Granada} basado en el software libre que además permita una metodología de desarrollo en el que continuamente se puedan añadir nuevas funcionalidades a la vez que se van testeando, para después de una fácil integración culminar con un despliegue automático. Este portal no almacenará los propios datos abiertos, si no que hará de presentación de los datos que estarán contenidos en otra plataforma destinada únicamente a dicho fin ({\tt OpenData UGR}).

\bigskip
Un resumen de los principales objetivos a alcanzar son:

\begin{itemize}
  \item \textbf{OBJ-1.} Promover la idea de que los desarrollos bajo software libre en general y en la administración pública en particular ayudan a generar confianza y demuestran compromiso con la transparencia.
  \item \textbf{OBJ-2.} Solucionar los problemas de generación de las tablas con los elementos de información.
  \item \textbf{OBJ-3.} Implementar en el portal {\tt UGR Transparente} una metodología de desarrollo continuo que se base en el uso de tests unitarios para cada una de las funcionalidades que se vayan añadiendo, test de cobertura que evalúen los test unitarios, integración continua ante cambios introducidos, despliegue automático de las actualizaciones que se vayan produciendo y aprovisionamiento software para la infraestructura.
\end{itemize}

Además como objetivo secundario tendremos:

\begin{itemize}
  \item \textbf{OBJ-4.} Estudiar la posibilidad de hacer una instalación totalmente personalizada del portal de datos de código abierto {\tt CKAN} para los datos abiertos de la \textbf{Universidad de Granada}, además de un aprovisionamiento que permitiera que esta instalación se pudiera realizar automáticamente en una infraestructura.
\end{itemize}

\bigskip
Destacar en los aspectos formativos previos más utilizados para el desarrollo del proyecto los conocimientos sobre infraestructuras virtuales para el tema de gestión de configuraciones, ingeniería de software para el análisis del proyecto e ingeniería de servidores para la realización de pruebas desde el aspecto hardware.

\section{Alcance de los objetivos}

La aplicación resultante visualmente será idéntica que la versión actual de {\tt UGR Transparente}, pero internamente habrá cambiado totalmente la metodología de desarrollo, pasando de un desarrollo sin control a uno totalmente controlado mediante pruebas unitarias e integración continua. Además los cambios ya no se tendrá que aplicar manualmente en el servidor, sino que se hará uso de un despliegue automático para tal fin.

\bigskip
Esta aplicación se usará de base en la \textbf{Universidad de Granada}, pero además el objetivo del desarrollo es que se pueda exportar para su uso en cualquier organización sin demasiada dificultad, de ahí que se vaya a estudiar también la posibilidad de personalización de {\tt CKAN}, ya que esto permitiría generar una gran plataforma de liberación de datos totalmente personalizable según el ámbito y las necesidades.

\section{Interdependencia de los objetivos}

Todos los objetivos son independientes entre sí, pero el primer objetivo (\textbf{OBJ-1}) es el principal motivador de este proyecto, por lo que aún sin representar el desarrollo de ningún trabajo en concreto es el que va a escudar y avalar el desarrollo de los otros. En aspectos más relacionados con la realización del proyecto, el segundo objetivo (\textbf{OBJ-2}) es el que se solucionará de forma inmediata ya que impide el correcto funcionamiento del portal. El resto de objetivos serán tratados en mayor medida durante los capítulos de implementación y pruebas.