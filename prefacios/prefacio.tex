\chapter*{}
%\thispagestyle{empty}
%\cleardoublepage

%\thispagestyle{empty}

%\cleardoublepage
%\thispagestyle{empty}

\begin{center}
{\large\bfseries \titulo}\\
\end{center}
\begin{center}
\autor\
\end{center}

\noindent{\textbf{Palabras clave}: software libre, transparencia, datos abiertos, sistema de control de versiones, 
aprovisionamiento, tests, integración continua, despliegue automático\\

\vspace{0.7cm}
\noindent{\textbf{Resumen}}\\

Desarrollar una plataforma para la transparencia y publicación de datos abiertos que sea desplegable en un infraestructura 
física o virtual basándose en el trabajo realizado en el portal UGR Transparente. Este desarrollo tendrá como objetivo que 
el resultado sea una plataforma totalmente basada en el software libre que se pudiera adaptar con facilidad a otro organismo, 
teniendo en cuenta funcionalidades y requisitos obligatorios, además de aspectos de accesibilidad y escalabilidad. 

\medskip
La herramientas básicas a usar serán un sistema de control de versiones y una plataforma de desarrollo colaborativo que 
albergue el proyecto, además usando dicho sistema de control de versiones. También se quiere que la plataforma se pueda ir 
desarrollando y administrando a la vez, por lo que para convertir todo esto es un proceso más ágil e ininterrumpido se usarán 
otras herramientas que permitan lo siguiente:

\begin{itemize}
  \item Realizar aprovisionamiento de las infraestructuras.
  \item Validación mediante tests unitarios.
  \item Comprobación de conflictos mediante integración continua.
  \item Actualizaciones mediante despliegue automático.
\end{itemize}

\cleardoublepage

\thispagestyle{empty}

\begin{center}
{\large\bfseries \tituloEng}\\
\end{center}
\begin{center}
\autor\\
\end{center}

%\vspace{0.7cm}
\noindent{\textbf{Keywords}: free software, transparency, opendata, version control system, provisioning, tests, continuous 
integration, automated deployment}\\

\vspace{0.7cm}
\noindent{\textbf{Abstract}}\\

Write here the abstract in English.

\chapter*{}
\thispagestyle{empty}

\noindent\rule[-1ex]{\textwidth}{2pt}\\[4.5ex]

Yo, \textbf{\autor}, alumno de la titulación \titulacion de la \textbf{\escuela\ de la \universidad}, con DNI XXXXXXXXX, 
autorizo la ubicación de la siguiente copia de mi Trabajo Fin de Grado en la biblioteca del centro para que pueda ser
consultada por las personas que lo deseen.

\vspace{6cm}

\noindent Fdo: \autor

\vspace{2cm}

\begin{flushright}
\ciudad, a \today
\end{flushright}

\chapter*{}
\thispagestyle{empty}

\noindent\rule[-1ex]{\textwidth}{2pt}\\[4.5ex]

D. \textbf{\tutor}, Profesor del Área de XXXX del Departamento de Arquitectura y Tecnología de Computadores de la \universidad.

\vspace{0.5cm}

\vspace{0.5cm}

\textbf{Informa:}

\vspace{0.5cm}

Que el presente trabajo, titulado \textit{\textbf{\titulo}}, ha sido realizado bajo su supervisión por \textbf{\autor}, y 
autorizo la defensa de dicho trabajo ante el tribunal que corresponda.

\vspace{0.5cm}

Y para que conste, expide y firma el presente informe en \ciudad\ a \today.

\vspace{1cm}

\textbf{El tutor:}

\vspace{5cm}

\noindent \textbf{\tutor}

\chapter*{Agradecimientos}
\thispagestyle{empty}

\vspace{1cm}

Poner aquí agradecimientos...

