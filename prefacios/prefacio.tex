\begin{center}
{\LARGE\bfseries\titulo}\\
\end{center}
\begin{center}
\autor\
\end{center}

\section*{Resumen}

\bigskip
\noindent{\textbf{Palabras clave}: software libre, transparencia, datos abiertos, sistema de control de versiones, 
aprovisionamiento, tests, integración continua, despliegue automático\\

Se pretende desarrollar una plataforma para la transparencia que sea desplegable en un infraestructura física o virtual 
basándose en el trabajo realizado en el portal UGR Transparente, respaldada por los datos abiertos publicados en la plataforma
OpenData UGR.

\bigskip
Este desarrollo tendrá como objetivo que el resultado sea una plataforma totalmente basada en el software libre que se 
pudiera adaptar con facilidad a otro organismo, teniendo en cuenta funcionalidades y requisitos obligatorios, además de 
aspectos de accesibilidad y escalabilidad. 

\bigskip
La herramientas básicas a usar serán un sistema de control de versiones y un sistema de desarrollo colaborativo que 
albergue el proyecto, aque además use dicho sistema de control de versiones. También se quiere que la plataforma se pueda 
desarrollarr y administrar simultáneamente, por lo que para convertir todo esto es un proceso más ágil e ininterrumpido se 
usarán otras herramientas que permitan lo siguiente:

\begin{itemize}
  \item Realizar aprovisionamiento de las infraestructuras.
  \item Validación mediante tests unitarios.
  \item Comprobación de conflictos mediante integración continua.
  \item Actualizaciones mediante despliegue automático.
\end{itemize}

\newpage
\begin{center}
{\LARGE\bfseries\tituloEng}\\
\end{center}
\begin{center}
\autor\
\end{center}

\section*{Extended abstract}

\bigskip
\noindent{\textbf{Keywords}: free software, transparency, opendata, version control system, provisioning, tests, continuous 
integration, automated deployment}\\

It is intended to develop a platform for transparency that is deployable on a physical or virtual infrastructure based on work
done on the site UGR Transparente, backed by open data published in the Open Data UGR.

\bigskip
This development aims that the result is a platform completely based on free software that could be easily adapted to another
organization, taking into consideration features and mandatory requirements, as well as issues of accessibility and scalability.
The platform which presents the data will develop in Node.js, which is a programming environment that operates at runtime based
on the Google's V8 JavaScript engine; also will use Express for web application development and Jade to generate HTML files
based on templates. Moreover, the platform that contains the data is based on CKAN, a open-source data portal platform
developed by The Open Knowledge Foundation, a not-for-profit organisation that promotes creating and sharing knowledge freely.

\bigskip
Being free software, this development is not limited to the working group that started the work, but rather is focused on that
anyone can make their contribution to the project; therefore, for all this can be handled is necessary a version control system,
in this case Git will be used, which is practically a standard in this area. In addition, it will also be used a recognized and
open collaborative based development platform as is GitHub, which is also integrated with Git, providing ease in the 
development and distribution of work, because any who access the project repository can freely copy it for its open license.

\bigskip
An important point to facilitate system administration is the provisioning. Provisioning a machine, as the name suggests, is to
provide to the machine all the resources needed for its performance, in our case we refer to all the software necessary for
that the platform developed works properly in such infrastructure. In the case of transparency portal, as the entire project
is hosted on GitHub, we can set up a utility like Ansible to download the project and then install on the machine all the
necessary software so that the platform can function properly. In contrast, such as the open data portal is not a own
development, their provisioning will consist of installing and customizing CKAN for the University of Granada.

\bigskip
During the development of any software application, new features are introduced to the software gradually, so must be ensured
that the new features do not compromise the overall stability of the project. For this purpose the unit tests are written, 
which could be considered as small programs within the system that handles check by assertions and behavior patterns that all
elements operate as they should. To verify that our unit tests are sufficient also need to pass a coverage test that tells us
that all the functionality of our platform are properly validated by the corresponding unit test. There are many libraries that
allow both actions, but for the facility of work between them, unit test will be passed with Mocha and coverage test will be
with Istanbul.

\bigskip
Once we have the tests written, we don't need concern ourselves with run them manually, we have the option of running in a
external platform every time we make changes and directly get the results, this is continuous integration. Continuous
integration will be made with Travis CI and the operation is very simple: every time we make a change in GitHub, Travis CI
download the latest version of the code, builds it and passes the tests we have written, to finish by returning the results
of the tests that will make us know whether changes have produced a conflict in the system. 

\bigskip
The last thing to keep in mind is about deploying the changes automatically on our server, can be a tedious procedure having to
manually access to the infrastructure of our platform every time that we want to apply the new changes we have made in the
application, so we'll use the tool Flightplan for automatic deployment; only with specify our server as the target we can set
a series of tasks that will make that automatically for the updates we're making become effective on the main machine.

\bigskip
By using these tools we are getting an application in which development and management are closely related and automated, which
makes the application much more reliable and easy recovery in case of problems.

\newpage
\thispagestyle{empty}
\
\vspace{3cm}

\noindent\rule[-1ex]{\textwidth}{2pt}\\[4.5ex]

Yo, \textbf{\autor}, alumno de la titulación \titulacion\ de la \textbf{\escuela\ de la \universidad}, autorizo la ubicación de
la siguiente copia de mi Trabajo Fin de Grado (\textit{\titulo}) en la biblioteca del centro para que pueda ser consultada por las personas que 
lo deseen.

\vspace{6cm}

\noindent Fdo: \autor

\vspace{2cm}

\begin{flushright}
\ciudad, a \today
\end{flushright}

\newpage
\thispagestyle{empty}
\
\vspace{3cm}

\noindent\rule[-1ex]{\textwidth}{2pt}\\[4.5ex]

D. \textbf{\tutor}, Profesor del Departamento de Arquitectura y Tecnología de Computadores de la \universidad.

\vspace{0.5cm}

\vspace{0.5cm}

\textbf{Informa:}

\vspace{0.5cm}

Que el presente trabajo, titulado \textit{\textbf{\titulo}}, ha sido realizado bajo su supervisión por \textbf{\autor}, y 
autoriza la defensa de dicho trabajo ante el tribunal que corresponda.

\vspace{0.5cm}

Y para que conste, expide y firma el presente informe en \ciudad\ a \today.

\vspace{1cm}

\textbf{El tutor:}

\vspace{5cm}

\noindent \textbf{\tutor}

\chapter*{Agradecimientos}
\thispagestyle{empty}

\vspace{1cm}

[Poner aquí agradecimientos]