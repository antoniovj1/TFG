\chapter{Introducción}

En la actualidad, nuestro país al igual que muchos otros se rige por lo que se conoce como \textbf{``democracia''}. Si buscamos
el significado de esta palabra, encontraremos que su definición es: 

\begin{quote}Sistema político que defiende la soberanía del pueblo y el derecho del pueblo a elegir y controlar a sus gobernantes.
\newline(\url{http://www.oxforddictionaries.com/es/definicion/espanol/democracia})
\end{quote}

Aunque \textit{``el derecho del pueblo a elegir''} es lo importante, para que esa elección pueda ser coherente, primero será el propio pueblo el que deberá tener la obligación a conocer lo que elige, y la única forma de conocer lo que se elige es a través de la transparencia. Por este motivo, el pueblo debe exigir transparencia en el funcionamiento de las instituciones públicas, y esas instituciones públicas siempre no tengan nada que esconder debería facilitar de buena fe (y no porque haya una ley que les obligue) todo la información que el ciudadano le requiera.

\bigskip
Desde que se aprobó la Ley de Transparencia en el año 2013, es obligatorio que todas las entidades públicas garanticen la transparencia en su actividad mediante la publicación de la información de la misma y faciliten el derecho de los ciudadanos al acceso de dicha información. Sin embargo, en el caso de las universidad públicas en particular, año y medio después de la aprobación de esta ley, podemos ver que son pocas las instituciones que tienen un portal de transparencia que realmente facilite datos útiles y cuya accesibilidad sea la adecuada para el acceso de todo tipo de población.

\bigskip
Estas inquietudes fueron las que hicieron en un primer momento que se comenzara con el desarrollo del portal de transparencia de la Universidad de Granada a principios del año pasado, cuyo desarrollo le fue encargado a la Oficina de Software Libre de la propia universidad y aproximadamente a los 7 meses de comenzar su desarrollo fue presentada una primera versión.

\bigskip
En febrero de este año entré como becario en la Oficina de Software Libre y me integré en el equipo de desarrollo del portal, siendo mis mayores contribuciones cambiar el origen de datos de la página para eliminar un error que se producía frecuentemente durante la navegación por el parte, y además, implantar una metodología de desarrollo DevOps. El funcionamiento de todos estos conceptos será ampliado en el capítulo 5 (Diseño) y en el capítulo 6 (Implementación), donde se explicará con más detalle en que consisten y como han sido integrados en el proyecto.

\bigskip
Además, antes de pasar a detalles tan técnicos, en los capítulos previos se explicarán otros aspectos del proyecto como son:

\begin{itemize}
  \item En el capítulo 2 (Objetivos), detallar de forma algo más concreta los objetivos determinados que se quieren cumplir con este proyecto.
  \item En el capítulo 3 (Planificación), la planificación y desarrollo de cada una de las fases del proyecto .
  \item En el capítulo 4 (Análisis), especificar todos los requisitos que se necesitan cubrir con el software a desarrollar, así como describir cómo esos requisitos tienen que tomar forma en el desarrollo.
\end{itemize}

\bigskip
Una vez estén realizadas tanto la parte descriptiva como la parte de desarrollo, quedarán solo por añadir los apartados finales; el capítulo 7 (Pruebas) en el que se explicarán las diferentes pruebas a las que ha sido sometido el software desarrollado para comprobar su correcto funcionamiento, y el capítulo 8 (Conclusiones y trabajos futuros) en el que expondrán los conocimientos y cuestiones sacados del desarrollo del proyecto, además de todo el trabajo que podría quedar pendiente para ampliar el portal.

\bigskip
[REFERENCIAR TAMBIÉN APÉNDICES]