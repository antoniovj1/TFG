\chapter{Especificación de requisitos}

\section{Objetivos}

El objetivo de este proyecto es el de obtener un portal de transparencia para la Universidad de Granada basado en el software 
libre que además permita una metodología de desarrollo en el que continuamente se puedan añadir nuevas funcionalidades a la 
vez que se van testeando, para después de una fácil integración culminar con un despliegue automático. Este portal no almacenará
los propios datos abiertos, si no que hará de presentación de los datos que estarán contenidos en otra plataforma destinada 
únicamente a dicho fin (OpenData UGR).

\bigskip
La aplicación final se usará en la Universidad de Granada, pero además el objetivo del desarrollo es que se pueda exportar 
para su uso en cualquier organización sin demasiada dificultad.

\bigskip
Un resumen de los principales objetivos a alcanzar son:

\begin{itemize}
  \item \textbf{OBJ-1.} La plataforma UGR Transparente enlazará los datos abiertos contenidos en la plataforma OpenData UGR
  (algo que ya hace, pero con problemas existentes que se deben solucionar).
  \item \textbf{OBJ-2.} El desarrollo de la plataforma UGR Transparente seguirá una metodología de desarrollo continuo que 
  contará con la implementación de tests unitarios para cada una de las funcionalidades que se vayan añadiendo, test de 
  cobertura que evalúen los test unitarios, integración continua ante cambios introducidos, despliegue automático de las
  actualizaciones que se vayan produciendo y aprovisionamiento software para la infraestructura.
  \item \textbf{OBJ-3.} Realizar una aprovisionamiento para la plataforma contenedora de los datos abiertos OpenData UGR que 
  permita una instalación con una base de CKAN más personalizada.
\end{itemize}
  
\section{Descripción de los usuarios finales}

El usuario de la aplicación será cualquier persona que tenga interés por conocer datos internos de la Universidad de Granada 
fácilmente. Como no se quiere enfocar en un público objetivo, el portal tiene que ser fácil de utilizar tanto para personas 
con experiencia en la navegación de páginas web como para las que no la tengan.

\section{Requisitos}

Todo software se desarrolla para cubrir una necesidad, por lo que en este apartado vamos a describir los requisitos que se 
estiman necesarios para cubrir los objetivos propuestos.

\subsection{Requisitos Funcionales}

Los requisitos funcionales son las características que tiene que implementar el sistema para cubrir todas las necesidades de 
los distintos usuarios. Al usuario lo único que le interesa es ver una página web estática con la información que desea 
consultar, por lo que el único requisito imprescindible es que cuando pulse una categoría esta se despliegue y cuando se 
pulse un enlace a OpenData UGR este nos lleve al conjunto de datos correspondiente. Además, se habilitará un buscador para que
también se pueda acceder a la información mediante la introducción de palabras clave. Todos los requisitos funcionales que 
tratará este proyecto están únicamente dirigidos a la plataforma UGR Transparente.

\begin{itemize}
  \item \textbf{RF-1.} Acceso a la información:
  \begin{itemize}
    \item \textbf{RF-1.1.} Al seleccionar una categoría esta se despliega y aparecen sus subcategorías.
    \item \textbf{RF-1.2.} Cuando se pulsa sobre el enlace de un elemento, automáticamente se accede a la información contenida 
    de ese elemento en OpenData UGR.
    \item \textbf{RF-1.3.} Para obtener información sobre un tema en concreto se introducirá el termino clave relacionado en 
    el buscador.
    \end{itemize}
\end{itemize}

\begin{itemize}
  \item \textbf{RF-2.} Presentación de la información:
  \begin{itemize}
    \item \textbf{RF-2.1.} Si el elemento de información no es un archivo PDF este se podrá previsualizar desde el mismo portal
    de transparencia sin necesidad de acceder a OpenData UGR.
    \item \textbf{RF-2.2.} Siempre se podrá descargar el archivo con la información contenido en OpenData UGR sobre cualquier 
    elemento visible.
  \end{itemize}
\end{itemize}
	
Los aspectos de funcionalidad ya se encuentran implementados de una fase previa al proyecto por lo que esta será la base de la
que se partirá para el desarrollo.

\subsection{Requisitos no Funcionales}

Los requisitos no funcionales son las características propias del desarrollo, pero que no tienen que estar relacionadas con su 
funcionalidad. En este caso nos referimos a todas las características que se requieren para que la aplicación siga un 
desarrollo ágil de despliegue continuo y administración automatica de la plataforma UGR Transparente.

\begin{itemize}
  \item \textbf{RN-1.} Procesar la información que será mostrada en las distintas páginas del portal.
  \item \textbf{RN-2.} Implementar tests unitarios para validar que todas las funcionalidades programadas funcionan 
  correctamente.
  \item \textbf{RN-3.} Realizar test de cobertura que compruebe la fiabilidad que proporcionan los tests unitarios.
  \item \textbf{RN-4.} Usar integración continua para asegurarse que los cambios introducidos no producen conflictos en la 
  plataforma.
  \item \textbf{RN-5.} Usar despliegue automático para actualizar con los nuevos cambios la plataforma una vez estos han sido 
  validados.
  \item \textbf{RN-6.} Elaborar una aprovisionamiento que permita que en una infraestructura por determinar se pueda instalar 
  automáticamente la plataforma y todos los elementos necesarios.
\end{itemize}

En cuanto a la plataforma OpenData UGR, este proyecto solo contempla un único objetivo no funcional que está relacionado con
la administración del sistema más que con que aspectos de desarrollo software.

\begin{itemize}
  \item \textbf{RN-7.} Crear una aprovisionamiento y personalización de CKAN para OpenData UGR.
\end{itemize}

\subsection{Requisitos de Información}

Los requisitos de información se refieren a la información que es necesaria almacenar en el sistema. La única información 
relevante que se va a almacenar son los datos descriptivos y de enlace de cada uno de los elementos del portal OpenData UGR 
que se van a mostrar en UGR Transparente.

\newpage
\begin{itemize}
  \item \textbf{RI-1.} Datos abiertos.
  \begin{itemize}
    \item Información sobre cada uno de los elementos que se van a mostrar en el portal de transparencia como datos abiertos.
    \item Contenido: nombre, categoría, conjunto de datos, enlace a OpenData UGR, enlace al recurso.
  \end{itemize}
\end{itemize}

\section{Modelos de casos de uso}

Aunque ya se ha indicado que la parte funcional ya se encuentra implementada de forma previa a este proyecto, se van a incluir
unos modelos de caso de uso simples para dar un visión más clara del funcionamiento general de la plataforma UGR Transparente.

\subsection{Jerarquía de casos de uso}

\subsection{Diagramas de modelos de casos de uso}

\subsubsection*{Diagrama de paquetes}

\subsubsection*{Diagrama de casos de uso}

Segundo formato.

\subsubsection*{Descripción básica de actores}

\subsubsection*{Descripción casos de uso}

\subsubsection*{Diagramas de actividad}

\section{Glosario de términos}

Poner aquí o en otro apartado más adelante.